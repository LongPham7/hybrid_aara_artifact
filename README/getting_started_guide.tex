% !TeX root = README.tex

\section{Getting Started Guide}

Our artifact, dubbed Hybrid Resource-Aware ML (RaML), is a program analysis tool
that takes in an OCaml program and infers its worst-case polynomial cost bound
using the technique Hybrid AARA described in the paper.

The artifact is wrapped inside the accompanying Docker image
\texttt{hybrid\_aara.tar.gz}.
%
Before running it, first install Docker as instructed here:
\url{https://docs.docker.com/engine/install/}.
%
To see if Docker has been installed properly, run
\begin{verbatim}
$ docker --version
Docker version 25.0.3, build 4debf41
\end{verbatim}

Load and run the Docker image as follows:
\begin{verbatim}
$ docker load --input hybrid_aara.tar.gz
$ docker run --name hybrid_aara -it --rm hybrid_aara
root@fca41f6f696d:/home/hybrid_aara/benchmark_suite/playground#
\end{verbatim}
%
It creates a Docker conainer (i.e., a runnable instance of the Docker image) and
starts a shell inside the container.
%
Throughout this document, any command line starting with \texttt{\#} is executed
inside the Docker container, and any command line starting with \texttt{\$} is
executed in your local machine's terminal.

The initial working directory \texttt{playground} has a file \texttt{append.ml},
which stores an OCaml function \texttt{append}.
%
Given two inputs lists, the \texttt{append} function appends the first input to
the second one.
%
For each element in the first input that the function processes, it incurs a
cost ranging from 0.5 to 1.0.
%
Therefore, the worst-case cost bound of the function is $1.0 \cdot M$, where $M$
is the length of the first input list.

Hybrid RaML's executable is the file \texttt{/home/hybrid\_aara/raml/main}.
%
Its location has already been added to the \texttt{PATH} variable so that we can
call Hybrid RaML by running \texttt{main} anywhere in the Docker container's
filesystem.
%
Hybrid AARA (and its implementation Hybrid RaML) provides three modes of hybrid
resource analysis: Hybrid \Opt{}, Hybrid \BayesWC{}, and Hybrid \BayesPC{}.

\subsection{Hybrid \Opt{}}

To perform Hybrid \Opt{} on the \texttt{append} function, run
\begin{verbatim}
# main stat_analyze ticks 1 -m append.ml append -config config_opt.json
\end{verbatim}
%
In the command line, \texttt{stat\_analyze} means we perform Hybrid AARA
(instead of the conventional AARA).
%
The argument \texttt{ticks} means the resource metric of our interest is the
tick metric (instead of heap memory, for example).
%
The argument 1 means we want to infer degree-1 polynomial cost bounds.
%
The argument \texttt{append.ml} specifies the input file name, and the argument
\texttt{append} specifies the function whose cost bound we want to infer.
%
Finally, the argument \texttt{config\_opt.json} is the configuration file for
Hybrid \Opt{}.

In the output printed on the stdout, you should find
\begin{verbatim}
== append :

[int list; int list] -> int list

Non-zero annotations of the argument:
    0.85  <--  ([::(*)], [])

Non-zero annotations of result:

Simplified bound:
    0.85*M
  where
    M is the number of ::-nodes of the 1st component of the argument
\end{verbatim}
%
It means that the cost bound of the function \texttt{append} inferred by Hybrid
\Opt{} is $0.85 \cdot M$, where $M$ is the first input's length.

\subsection{Hybrid \BayesWC{}}

Next, to perform Hybrid \BayesWC{}, run
\begin{verbatim}
# main stat_analyze ticks 1 -m append.ml append -config config_bayeswc.json
\end{verbatim}
where the configuration file is now \texttt{config\_bayeswc.json}, which
contains hyperparameters for Bayesian inference.
%
The analysis should finish in 50 seconds.
%
Its output on the stdout should be close to
\begin{verbatim}
== append :

  [int list; int list] -> int list

  Non-zero annotations of the argument:
(avg, std) = (   2.18047,   0.633256)  <--
                                         ([::(*)], [])
(avg, std) = ( 0.0101388,  0.0158341)  <--  ([], [::(*)])
(avg, std) = (         0,          0)  <--  ([], [])

  Non-zero annotations of result:
(avg, std) = (         0,          0)  <--
                                   [::(*)]
(avg, std) = (         0,          0)  <--  []

  Simplified bound:
     0.0101388*L + 2.18047*M
   where
     L is the number of ::-nodes of the 2nd component of the argument
     M is the number of ::-nodes of the 1st component of the argument
\end{verbatim}
%
Unlike Hybrid Opt{} that infers a single cost bound, Hybrid \BayesWC{} returns a
collection of cost bounds, which approximate their posterior distribution in
Bayesian inference.
%
In the stdout, 0.0101388 is the average degree-one coefficient for the first
input in the posterior distribution, and 2.18047 is the average
coefficient for the second input.

\subsection{Hybrid \BayesPC{}}

To perform Hybrid \BayesPC{}, run
\begin{verbatim}
# main stat_analyze ticks 1 -m append.ml append -config config_bayespc.json
\end{verbatim}
using the configuration file \texttt{config\_bayespc.json}.
%
The analysis should terminate within 90 seconds.
%
The output printed on the stdout should be close to
\begin{verbatim}
== append :

  [int list; int list] -> int list

  Non-zero annotations of the argument:
(avg, std) = (   2.03295,   0.228391)  <--
                                         ([::(*)], [])
(avg, std) = (  0.281939,   0.146543)  <--  ([], [::(*)])
(avg, std) = (  0.164309,  0.0824471)  <--  ([], [])

  Non-zero annotations of result:
(avg, std) = (         0,          0)  <--
                                   [::(*)]
(avg, std) = (         0,          0)  <--  []

  Simplified bound:
     0.164309 + 0.281939*L + 2.03295*M
   where
     L is the number of ::-nodes of the 2nd component of the argument
     M is the number of ::-nodes of the 1st component of the argument
\end{verbatim}
where each coefficient in the inference result \texttt{0.164309 + 0.281939*L +
  2.03295*M} is the average of its posterior distribution.
