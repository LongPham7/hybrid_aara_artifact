% !TeX root = README.tex

\section{Step-by-Step Instructions}

\subsection{Benchmark Suite}

The artifact contains 10 benchmark programs for Hybrid AARA.
%
As listed in Section 7 of the paper, these benchmarks are: \texttt{MapAppend},
\texttt{Concat}, \texttt{InsertionSort2}, \texttt{QuickSort},
\texttt{QuickSelect}, \texttt{MedianOfMedians}, \texttt{ZAlgorithm},
\texttt{BubbleSort}, \texttt{Round}, and \texttt{EvenOddTail}.
%
Inside the artifact, these benchmarks are referred to as \texttt{append},
\texttt{concat}, \texttt{insertion\_sort}, \texttt{quicksort},
\texttt{quickselect}, \texttt{linear\_select}, \texttt{z\_algorithm},
\texttt{bubble\_sort}, \texttt{round}, and \texttt{even\_split\_odd\_tail}.
%
We have tested both data-driven resource analysis and hybrid resource analysis
on the first seven benchmarks, and only data-driven resource analysis on the
remaining three benchmarks.

Inside the benchmark directory \texttt{/home/hybrid\_aara/benchmark\_suite},
each benchmark has its own directory.
%
Furthermore, inside the benchmarks' directories, we have the following
directories:
\begin{enumerate}
  \item \texttt{utility}: Python source files that contain (i)
        benchmark-specific configurations for hybrid resource analysis and (ii)
        functions for analyzing the inference result.
  \item \texttt{bin}: input and output files for Hybrid RaML.
  \item \texttt{images}: plots of inferred cost bounds.
\end{enumerate}

For each benchmark, the directory \texttt{bin} already contains the experiment
results that we reported in the paper.
%
We will demonstrate how to generate the experiment results from scratch in
\cref{sec:generate experiment results}.

\subsection{Display Experiment Results}

We demonstrate how to create tables and plots in the paper from the experiment
results that are stored in each benchmark's \texttt{bin} directories.

\paragraph{Table~1}

In Table~1 in the paper, the 4\textsuperscript{th} and 5\textsuperscript{th}
columns display the proportions of sound cost bounds inferred by data-driven and
hybrid resource analyses, respectively.
%
To display these two columns, run
\begin{verbatim}
# cd /home/hybrid_aara/benchmark_suite/toolbox
# python3 run_analysis.py soundness
\end{verbatim}

The 6\textsuperscript{th} and 7\textsuperscript{th} columns of Table~1 display
the analysis time of data-driven hybrid resource analyses, respectively.
To display these two columns, run
\begin{verbatim}
# cd /home/hybrid_aara/benchmark_suite/toolbox
# python3 run_analysis.py time
\end{verbatim}

\paragraph{Figure~5}

Figure~5 in the paper displays relative errors of 5 selected benchmarks.
%
To create this plot, run
\begin{verbatim}
# cd /home/hybrid_aara/benchmark_suite/toolbox
# python3 visualization_relative_errors.py
\end{verbatim}
%
This command creates a plot and stores it as a file
\texttt{relative\_errors.pdf} inside the directory
\texttt{/home/hybrid\_aara/benchmark\_suite/images}.
%
To view the plot, we first transfer the plot from the Docker container to the
local filesystem.
%
Run the following command in the terminal of your local machine:
\begin{verbatim}
$ docker cp hybrid_aara:/home/hybrid_aara/benchmark_suite/images/relative_errors.pdf 
    <path_to_local_directory>
\end{verbatim}
where the second argument \texttt{<path\_to\_local\_directory>} is the
destination directory on your local filesystem.
%
You can then view it on your local machine.

\subsection{Generate Experiment Results}
\label{sec:generate experiment results}
